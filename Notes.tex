\documentclass[oneside]{book}

\usepackage{Style/Master}
\usepackage{Style/boxes}
\usepackage{Style/DefNoteFact}
\usepackage{Style/QnsProof}
\usepackage{Style/Thms}
\usepackage{Style/Env}

\usetikzlibrary{fit}
\newcommand\addvmargin[1]{
  \node[fit=(current bounding box),inner ysep=#1,inner xsep=0]{};
}

\begin{document}

\begin{titlepage}
\frontmatter
\begin{tikzpicture}[font=\sffamily,remember picture,overlay]
    \node[fill=Goldenrod,anchor=north, minimum width=\paperwidth, minimum height=2cm] (names)
 at ([yshift=-3cm]current page.north) {
  \begin{tabular}{c}
      \\
      \color{blue} {\fontsize{24.88pt}{65pt}\selectfont A-Levels Math Notes}\\[2mm]
      \Large \color{blue}{Grass}\\
      \\
  \end{tabular}
  };
\end{tikzpicture}

\begingroup
\let\clearpage\relax
\vspace{5cm}
\tableofcontents
\endgroup
\mainmatter
\end{titlepage}

\chapter{A1: Inequalities and Equations}
\section{Solving Inequalities}
\subsection{Rational Inequalities}
\begin{stbox}{General Methods}
  \begin{enumerate}
    \item Quadratic formula for factorisation / finding roots (of polynomial).
    \item Completing the square.
    \item Discriminant/Completing the Sq to eliminate{\hyperlink{INEQGM3}{*}} factors which are \emph{always} positive or negative (e.g. removing \(x^2-3x+4\)). \emph{Note to include coefficient of \(x^2\) in argument.}
    \item GC (include sketch).
    \item \emph{Rational Functions}\footnote{Fractions of Polynomials}: Move everything to one side (\(+\),\(-\)), then use number line.
    \item Number line (more complicated functions).
\end{enumerate}
\end{stbox}
\begin{IN}
    \begin{itemize}[label=\footnotesize \(\square\) \normalsize]
        \item \hypertarget{INEQGM3}{Eliminating Factors --- \emph{only}\footnote{Counterexample: \(P(x)=x(3x^2-9x+10) \leq 2\) iff \(x \leq 2\) is false. E.g.: \(P(1.8)=6.336 \not\leq 2\).} works for \(c=0\) in \(f(x) \geq c\) or \(f(x) \leq c\).}
        \item Discriminant --- include coefficient of \(x^2\) in argument.
        \item Rational functions --- exclude the values that causes division by zero to occur.
        \item With inequalities, be really careful about multiplication! If \(x>y\) \emph{and} \hly{\(z>0\)}, then \(xz>yz\). 
        \item Cross multiplication preserves order for \(\frac{x}{y}<\frac{x'}{y'}\) iff \(y\) and \(y'\) are \emph{both} positive or negative.\footnote{Otherwise, note the counterexample \(\frac{1}{2}<\frac{1}{-3}\).}
        \item Squaring preserves order for \(x<y\) iff \(x\) and \(y\) are \emph{both} positive or negative.
    \end{itemize}
\end{IN}
\section{Modulus Inequalities}
\begin{fact}
    Given \(x \in \mathbb{R}\), we have that 
    \begin{itemize}
        \item \(\lvert x \rvert \geq 0\),
        \item \(\lvert x^2 \rvert=\lvert x \rvert^2=x^2\),
        \item \(\sqrt{x^2}=\lvert x \rvert\).
    \end{itemize}
    And as long as \(x \in \mathbb{R}^+\),
    \begin{itemize}
        \item \(\sqrt{x}^2=\lvert x \rvert\).
    \end{itemize}
\end{fact}
\begin{stbox}{Useful Properties}
  For every \(x,k \in \mathbb{R}\):
  \begin{enumerate}[label=(\alph*)]
      \item \(\lvert x \rvert < k\) iff\footnote{Notice that \(k>0\) here since \(0 \leq \lvert x \rvert < k\).} \(-k<x<k\).
      \item \(\lvert x \rvert > k\) iff \(x<-k\) or \(x>k\).
  \end{enumerate}
  \footnotesize (of course, similarly applies for the non-strict ordering \(\leq\)) \normalsize
\end{stbox}
\begin{IN}
    \begin{itemize}[label=\(\circ\)]
        \item Note that when solving for \(\lvert x \rvert=y\), \(\lvert x \rvert < y\), etc, \(y\) must be greater than or equal to 0. In other words, there may be solutions we will need to reject.\\ \footnotesize(For \(<\), equality is of course not allowed.)\normalsize  
    \end{itemize}
\end{IN}
\begin{IN}
\begin{itemize}[label=\scriptsize\(\triangle\)\normalsize]
  \item Carelessness: Look at the question carefully! If they ask for a \emph{set} of values, then rmb to give it as a \emph{set}!
  \item Exponentiation and Logarithms: Simply use \(\ln\) and avoid \(\log_c\) for \(c<1\).\footnote{Order is \emph{Preserved} under exponentiation/logarithms if the base is \emph{larger than} one. Otherwise, when it is \emph{less than} one, the order is \emph{reversed}. \url{https://www.desmos.com/calculator/gd8z5fa0bg}}
\end{itemize}
\end{IN}

\section{System of Linear Equations}
\begin{stbox}{Things}
  \begin{itemize}[label=\(\chi\)]
    \item  For more complicated real-world-context qns, try playing around with the values (e.g. use simult eqns) first. It may work out nicer than expected.
  \end{itemize}
\end{stbox}
\section{Summary}

\begin{lbox}[colbacktitle=white, coltitle=black, colframe=black]{G.C. Skills} 
  \begin{enumerate}
    \item Plotting curves \(y=f(x)\) in G.C.
    \item How to use simultaneous equation solver.
  \end{enumerate}
\end{lbox}
\begin{IN}
  \begin{itemize}[label=\footnotesize \(\qed\) \normalsize]
    \item \hypertarget{INEQGM3}{Eliminating Factors --- \emph{only}\footnote{Counterexample: \(P(x)=x(3x^2-9x+10) \leq 2\) iff \(x \leq 2\) is false. E.g.: \(P(1.8)=6.336 \not\leq 2\).} works for \(c=0\) in \(f(x) \geq c\) or \(f(x) \leq c\).}
    \item Discriminant --- include coefficient of \(x^2\) in argument.
    \item Rational functions --- exclude the values that causes division by zero to occur.
    \item With inequalities, be really careful about multiplication! If \(x>y\) \emph{and} \hly{\(z>0\)}, then \(xz>yz\). 
    \item Cross multiplication preserves order for \(\frac{x}{y}<\frac{x'}{y'}\) iff \(y\) and \(y'\) are \emph{both} positive or negative.\footnote{Otherwise, note the counterexample \(\frac{1}{2}<\frac{1}{-3}\).}
    \item Squaring preserves order for \(x<y\) iff \(x\) and \(y\) are \emph{both} positive or negative.
    \item  Note that when solving for \(\lvert x \rvert=y\), \(\lvert x \rvert < y\), etc, \(y\) must be greater than or equal to 0. In other words, there may be solutions we will need to reject.\\ \footnotesize(For \(<\), equality is of course not allowed.)
    \item Carelessness: Look at the question carefully! If they ask for a \emph{set} of values, then rmb to give it as a \emph{set}!
    \item Exponentiation and Logarithms: Simply use \(\ln\) and avoid \(\log_c\) for \(c<1\).\footnote{Order is \emph{Preserved} under exponentiation/logarithms if the base is \emph{larger than} one. Otherwise, when it is \emph{less than} one, the order is \emph{reversed}. \url{https://www.desmos.com/calculator/gd8z5fa0bg}}
    \item  For more complicated real-world-context qns, try playing around with the values (e.g. use simult eqns) first. It may work out nicer than expected.
\end{itemize}
\end{IN}

\chapter{A5.1: Differentiation}
\section{Limits}
\section{Geometrical Results of the Derivatives}
\begin{definition*}{}{}
  \begin{enumerate}[label=(\roman*)]
    \item A function \(f\) is called (strictly) increasing on an interval \(I\) iff \(f'(x)>0\) for all \(x \in I\).
    \item A function \(f\) is called monotonically increasing on an interval \(I\) iff \(f'(x) \geq 0\) for any \(x \in I\).
  \end{enumerate}
\end{definition*}
\begin{stbox}{Things To Know}
  \begin{enumerate}
    \item How to sketch the graph of the integral or\footnote{Of course, provided that \(f\) is integrable/differentiable.} derivative of a function \(f\).
    \item Relationship btw. a function \(f\) and its derivative, \(f'\):\\[4mm]
    \begin{tabular}{|c|c|}
      \hline
      \(y=f(x)\) & \(y=f'(x)\)\\
      \hline
      Vertical asymptote at \(x=a\) & Vertical asymptote at \(x=a\).\\
      \hline
      Horizontal asymptote at \(y=a\) & Horizontal asymptote \(y=0\).\\
      \hline
    \end{tabular}
  \end{enumerate}
\end{stbox}

\chapter{B1(A): Graphing Techniques (Part I)}
\section{Graphing `Familiar' Functions and Asymptotic bois}

\begin{definition*}{}{}
  \begin{enumerate}
    \item \textbf{Lines of Symmetry}: A \emph{line of symmetry} of a function is a line, such that the function is a reflection of itself about that line.
    \item \textbf{Horizontal Asymptotes}: A (horizontal) line \(g(x)=c\) is the \emph{horizontal asymptote} of the curve \(f(x)\) iff \(\lim_{x \to \infty}{f(x)}=c\) (or with \(-\infty\) instead of \(\infty\)).\footnote{Otherwise notated by \(f(x) \to c\) as \(x \to \infty\).}
    \item \textbf{Vertical Asymptotes}: A (vertical) line \(x=c\) is a \emph{vertical asymptote} of the curve \(f(x)\) iff \(\lim_{x \to c}{f(x)}=\operatorname{\infty} \text{ or } -\infty\).
    \item \textbf{Oblique Asymptotes}: A line \(g(x)=mx+c\) --- where \(m \neq 0\) --- is an \emph{oblique asymptote} of the curve \(f(x)\) iff \(\lim_{x \to \infty}[f(x)-g(x)]=0\) (or with \(-\infty\) instead of \(\infty\)).
\end{enumerate}
\end{definition*}
\begin{stbox}{Curve Sketching (Rational Funcs)}
  \begin{enumerate}
    \item[\textbf{S}] Stationary points
    \item[\textbf{I}] Intersection with axes
    \item[\textbf{A}] Asymptotes   
  \end{enumerate}
  \begin{enumerate}[label=\roman*]
    \item Know how to sketch the graphs of \(y=\frac{ax+b}{cx+d}\) and \(y=\frac{ax^2+bx+c}{dx+e}\).
    \item Rectangular Hyperbolas \(\left(\text{of the form \(y=\frac{ax+b}{cx+d}\)}\right)\):
    \begin{itemize}
      \item \emph{Two} asymptotes, namely \(x=-\frac{d}{c}\) and \(y=\frac{a}{c}\).
      \item \emph{Two} lines of symmetry with gradients \(\pm 1\) \emph{and} pass through the intersection point of the aforementioned two asymptotes.
    \end{itemize}
    \item \emph{If} \(n=\operatorname{deg}P=\operatorname{deg}Q\), then
    \begin{itemize}
      \item \(y=R(x)\) is the \emph{horizontal} asymptote of \(\frac{P(x)}{Q(x)}=R(x)+\frac{S(x)}{Q(x)}\).
      \item Equivalently, \(y=\frac{\operatorname{coeff}_P(x^n)}{\operatorname{coeff}_Q(x^n)}\) is a \emph{horizontal} asymptote.\footnote{E.g.: \(y=\frac{1}{15}\) is a horizontal asymptote of \(y=\frac{\text{\hly{\(1\)}}x^2+2x-3}{(\text{\hly{\(5\)}}x+1)(\text{\hly{\(3\)}}x+2)}\).}
    \end{itemize}
    \item If \(\operatorname{deg}P=\operatorname{deg}Q+1\), then \(R(x)\) is an \emph{oblique} asymptote of \(\frac{P(x)}{Q(x)}=R(x)+\frac{S(x)}{Q(x)}\).
    \item Write down asymptotes and lines of symmetry.\footnote{E.g.: \begin{itemize}
      \item[] Asymptotes: \(x=4\), \(y=20\).
      \item[] Lines of Symmetry: \(y=x+16\), \(y=-x+24\).
    \end{itemize} } If none are present indicate with ``No lines of symmetry.''
  \end{enumerate}
\end{stbox}
\begin{IN}
  \begin{itemize}[label=---]
    \item \emph{Can} explicitly write out the asymptotes and lines of symmetry (or if they are not present) to be safe.
    \item Using the discriminant intelligently can result in nice answers.
  \end{itemize}
\end{IN}

\section{Conics}
\subsection{Ellipses}
``Tikz is pain, PGFPlots is suffering'' --- Wise Man.
\begin{center}
  \small
  \begin{tabular}{|c|c|c|}
    \hline
    & Ellipses & Hyperbolas\\
    \hline
    Standard Forms & \(\frac{(x-h)^2}{a^2}+\frac{(y-k)^2}{b^2}=1\) & 
    \begin{tabular}{@{}c@{}} 
     \\
    \(\frac{(x-h)^2}{a^2}-\frac{(y-k)^2}{b^2}=1\)\\
    \(\frac{(y-k)^2}{b^2}-\frac{(x-h)^2}{a^2}=1\)\\
    \\
    \end{tabular}\\
    \hline
    General Equation & 
    \begin{tabular}{@{}c@{}} 
      \(ax^2+by^2+cx^2+dx+e=0\),\\
      \footnotesize where \(\operatorname{sgn}(a)=\operatorname{sgn}b\). \normalsize
     \end{tabular}
     &
     \begin{tabular}{@{}c@{}} 
      \(ax^2+by^2+cx^2+dex+e=0\),\\
      \footnotesize where \(\operatorname{sgn}(a) \neq \operatorname{sgn}b\). \normalsize
     \end{tabular}\\
     \hline
     Center & \((h,k)\) & \((h,k)\)\\
    \hline
    \begin{tabular}{@{}c@{}} 
      Vertical `Radius'\\
      \footnotesize (variables here from \emph{standard form}!) \normalsize
    \end{tabular}
      & \(b\) & \(b\)\\
      \hline
    \begin{tabular}{@{}c@{}} 
        Horizontal `Radius'\\
        \footnotesize (variables here from \emph{standard form}!) \normalsize
    \end{tabular}
    & \(a\) & \(a\)\\
    \hline
    \begin{tabular}{@{}c@{}} 
      Vertical  Vertices\\
      \footnotesize (variables here from \emph{standard form}!) \normalsize
    \end{tabular}
    & \((h, k \pm b)\) & \((h, k  \pm b)\)\\
    \hline
    \begin{tabular}{@{}c@{}} 
      Horizontal Vertices\\
      \footnotesize (variables here from \emph{standard form}!) \normalsize
    \end{tabular}
    & \((h \pm a,k)\) & \((h \pm a, k)\)\\
    \hline
    Shape & 
    \begin{tikzpicture}[scale=0.5]
      
      \begin{axis}[axis lines=middle,axis line style =-{Classical TikZ Rightarrow[length=5pt 3 0]},every axis x label/.style = {%
        at = {(xticklabel cs:1.05)},
        anchor = north},
      every axis y label/.style = {%
        at = {(yticklabel cs:1.05)},
        anchor=east},
        xtick=\empty, ytick=\empty,clip=false,xmin=0,xmax=11,xlabel=\Large\(x\),ylabel=\Large\(y\),ymin=0,ymax=6
        ]
        
      \draw[blue] (5,3) ellipse (5 and 2);
    
      % label center
      \node at (5,3) [below right] {\Large \((h,k)\)};

      \node at (5,3) {\Large \color{red}{\(\times\)}};
    
      % draw h-line
      \draw[->,arrows = -{Classical TikZ Rightarrow[length=5pt 3 0]}] (5,3) -- (0,3) node[midway, above] {\Large \(h\)};
    
      % draw another h-line
      \draw[->,arrows = -{Classical TikZ Rightarrow[length=5pt 3 0]}] (5,3) -- (10,3);
    
      % draw k-line
      \draw[->,arrows = -{Classical TikZ Rightarrow[length=5pt 3 0]}] (5,3) -- (5,5) node[midway, right] {\Large \(k\)};
    
      % draw another k-line
      \draw[->,arrows = -{Classical TikZ Rightarrow[length=5pt 3 0]}] (5,3) -- (5,1);

      \addplot+[
  mark=x,
  only marks,
  mark size=6pt,
  mark options={line width=1.5pt,red}
] 
  coordinates
  {(5,3)};
  
      \end{axis}
      % draw ellipse
      \addvmargin{3mm}
    \end{tikzpicture} & 
    \begin{tabular}{@{}c@{}} 
      \(\operatorname{coeff}(x^2)<0\)\\
      \begin{tikzpicture}[scale=0.5]
      
        \begin{axis}[axis lines=middle,axis line style =-{Classical TikZ Rightarrow[length=5pt 3 0]},every axis x label/.style = {%
          at = {(xticklabel cs:1.05)},
          anchor = north},
        every axis y label/.style = {%
          at = {(yticklabel cs:1.05)},
          anchor=east},
          xtick=\empty, ytick=\empty,clip=false,xmin=-1,xmax=11,xlabel=\Large\(x\),ylabel=\Large\(y\),ymin=-1,ymax=7
          ]
          
      \addplot [red,thick,domain=-2:2] ({sinh(x)+5}, {cosh(x)+3});
      \addplot [red,thick,domain=-2:2] ({-sinh(x)+5}, {-cosh(x)+3});
      \addplot[red,dashed,domain=1:9] {x-2};
      \addplot[red,dashed,domain=1:9] {-x+8};

        \node at (5,3) [below] {\Large \((h,k)\)};
  
        \node at (5,3) {\LARGE  \color{blue}{\(\times\)}};
      
        % draw h-line
        \draw[->,arrows = -{Classical TikZ Rightarrow[length=3pt 3 0]}] (5,4) -- (4,4) node[midway, above] {\Large \(h\)};
      
        % draw k-line
        \draw[->,arrows = -{Classical TikZ Rightarrow[length=3pt 3 0]}] (5,3) -- (5,4) node[midway, right] {\Large \(k\)};
        
        \addplot+[
  mark=x,
  only marks,
  mark size=6pt,
  mark options={line width=1.5pt}
] 
  coordinates
  {(5,3)};

        \end{axis}
        % draw ellipse
      \end{tikzpicture}\\
      \(\operatorname{coeff}(y^2)<0\)\\
      \begin{tikzpicture}[scale=0.5]
      
        \begin{axis}[axis lines=middle,axis line style =-{Classical TikZ Rightarrow[length=5pt 3 0]},every axis x label/.style = {%
          at = {(xticklabel cs:1.05)},
          anchor = north},
        every axis y label/.style = {%
          at = {(yticklabel cs:1.05)},
          anchor=east},
          xtick=\empty, ytick=\empty,clip=false,xmin=-1,xmax=11,xlabel=\Large\(x\),ylabel=\Large\(y\),ymin=-1,ymax=7
          ]
          
        \addplot [red,thick,domain=-2:2] ({cosh(x)+5}, {sinh(x)+3});
      \addplot [red,thick,domain=-2:2] ({-cosh(x)+5}, {sinh(x)+3});
      \addplot[red,dashed,domain=1:9] {x-2};
      \addplot[red,dashed,domain=1:9] {-x+8};

        \node at (5,3) [below=0.25cm] {\Large \((h,k)\)};
  
        \node at (5,3) {\LARGE  \color{blue}{\(\times\)}};
      
        % draw h-line
        \draw[->,arrows = -{Classical TikZ Rightarrow[length=3pt 3 0]}] (5,3) -- (4,3) node[midway, above] {\Large \(h\)};
      
        % draw k-line
        \draw[->,arrows = -{Classical TikZ Rightarrow[length=3pt 3 0]}] (4,3) -- (4,4) node[midway, left] {\Large \(k\)};

        \addplot+[
  mark=x,
  only marks,
  mark size=6pt,
  mark options={line width=1.5pt}
] 
  coordinates
  {(5,3)};

        \end{axis}
        % draw ellipse
      \end{tikzpicture}
    \end{tabular}\\
    \hline
    \begin{tabular}{@{}c@{}} 
      Asymptotes\\
      (No need to rmb!)
    \end{tabular}
    & \(y=k \pm \frac{b(x-h)}{a}\) & \(y=k \pm \frac{b(x-h)}{a}\)\\
    \hline
    Lines of Symmetry & \(x=h\), \(y=k\) & \(x=h\), \(y=k\)\\
    \hline
  \end{tabular}
  \normalsize
\end{center}
\begin{stbox}{General Info}
  \begin{itemize}[label=\(\mathscr{H}\)]
    \item To find asymptote of hyperbolas, just solve  
    \[\frac{(x-h)^2}{a^2}=\frac{(y-k)^2}{b^2}.\]
    \item Label vertices \emph{or} radii, together with the center and asymptotes.
  \end{itemize}
\end{stbox}
\section{Parametric Equations}
\begin{IN}
  \begin{itemize}[label=\(\star\)]
    \item Check the qns for any \emph{restrictions} on the parameter! And modify that of the G.C.'s accordingly (Tmin \& Tmax).
  \end{itemize}
\end{IN}
\section{Summary}
\begin{lbox}[colbacktitle=white, coltitle=black, colframe=black]{G.C. Skills} 
  \begin{enumerate}
    \item Plot conics with the two ways.
    \item Know G.C. functions like finding axial intercepts.
  \end{enumerate}
\end{lbox}
\begin{IN}
  \begin{itemize}[label=-\(\square\)-]
    \item \emph{Can} explicitly write out the asymptotes and lines of symmetry (or if they are not present) to be safe.
    \item Using the discriminant intelligently can result in nice answers.
    \item Check the qns for any \emph{restrictions} on the parameter! And modify that of the G.C.'s accordingly (Tmin \& Tmax).
  \end{itemize}
\end{IN}

\chapter{Statistics 1: Permutations and Combinations}
\section{The Addition and Multiplication Principles}
\begin{example}{The Addition Principle}{}
  There are three distinct cups of black sugar bubble tea and five unique cups of zero sugar bubbles tea available, I am buying \emph{exactly} one of them. How many choices do I have? Answer: \(3+5=8\).
\end{example}
\begin{example}{The Multiplication Principle}{}
  A college planning committee consists of 3 freshmen, 4 sophomores, 5 juniors, and 2 seniors. A subcommittee of 4, consisting of 1 person from each class, is to be chosen. How many different subcommittees are possible. Answer: \(3 \cdot 4 \cdot 5 \cdot 2=120\). 
\end{example}
\section{Permutation}
\begin{definition}{\(\perm{k}\)}{}
  \[\perm{k} := \frac{n!}{(n-k)!}\quad\]
\end{definition}

\end{document}