\documentclass[oneside]{book}
\usepackage[utf8]{inputenc}
\makeatletter
\DeclareRobustCommand{\rvdots}{%
  \vbox{
    \baselineskip4\p@\lineskiplimit\z@
    \kern-\p@
    \hbox{.}\hbox{.}\hbox{.}
  }}
\makeatother
\usepackage[outline]{contour}
\usepackage{environ}
\usepackage[dvipsnames]{xcolor}
\usepackage{tikz}
\usetikzlibrary {arrows.meta}
\usetikzlibrary{cd}
\NewEnviron{elaboration}{
\par
\begin{tikzpicture}
\node[rectangle,minimum width=\linewidth] (m) {\begin{minipage}{\linewidth}\BODY\end{minipage}};
\draw[dashed] (m.south west) rectangle (m.north east);
\end{tikzpicture}
}
\usepackage{cancel}
\usepackage{stackengine}
\newcommand\lbr{%
  \ensurestackMath{\stackengine{0pt}{\lfloor}{\lceil}{O}{c}{F}{F}{L}}}
\newcommand\rbr{%
  \ensurestackMath{\stackengine{0pt}{\rfloor}{\rceil}{O}{c}{F}{F}{L}}}
\usepackage[normalem]{ulem}
\newcommand\hlg{\bgroup\markoverwith
  {\textcolor{green}{\rule[-.5ex]{2pt}{2.5ex}}}\ULon}
\usepackage{mathrsfs}  
\usepackage{amsmath}
\makeatletter
\newcommand{\leqnomode}{\tagsleft@true}
\newcommand{\reqnomode}{\tagsleft@false}
\makeatother
\makeatletter
\newcommand{\ineq}{\mathrel{\text{\in@eq}}}
\newcommand{\in@eq}{%
  \oalign{%
    \hidewidth$\m@th\in$\hidewidth\cr
    \noalign{\nointerlineskip\kern1ex}%
    $\m@th\smash{-}$\cr
    \noalign{\nointerlineskip\kern-.5ex}%
  }%
}
\newcommand{\notineq}{\mathrel{\text{\notin@eq}}}
\newcommand{\notin@eq}{%
  \oalign{%
    \hidewidth$\m@th\notin$\hidewidth\cr
    \noalign{\nointerlineskip\kern1ex}%
    $\m@th\smash{-}$\cr
    \noalign{\nointerlineskip\kern-.5ex}%
  }%
}
\makeatother
\usepackage[many]{tcolorbox}
\usepackage{mathtools}
\usepackage{amssymb}
\usepackage{amsfonts}
\usepackage{amsthm}
\newtheorem{theorem}{Theorem}[section]
\tcolorboxenvironment{theorem}{
  colback=orange!15,
  boxrule=1pt,
  boxsep=1pt,
  left=2pt,right=2pt,top=2pt,bottom=2pt,
  oversize=2pt,
  colframe=blue!75!black,
  before skip=\topsep,
  after skip=\topsep,
}
\renewcommand{\thetheorem}{\arabic{section}\Alph{theorem}}
\newtheorem{corollary}[theorem]{Corollary}
\tcolorboxenvironment{corollary}{
  colback=pink!25,
  boxrule=1pt,
  boxsep=1pt,
  left=2pt,right=2pt,top=2pt,bottom=2pt,
  oversize=2pt,
  colframe=orange!15,
  before skip=\topsep,
  after skip=\topsep,
}
\newtheorem{lemma}[theorem]{Lemma}
\tcolorboxenvironment{lemma}{
  colback=yellow!15,
  boxrule=1pt,
  boxsep=1pt,
  left=2pt,right=2pt,top=2pt,bottom=2pt,
  oversize=2pt,
  colframe=orange!15,
  before skip=\topsep,
  after skip=\topsep,
}
\theoremstyle{definition}
\newtheorem*{definition}{Definition}
\tcolorboxenvironment{definition}{
  colback=black!5,
  boxrule=1pt,
  boxsep=1pt,
  left=2pt,right=2pt,top=2pt,bottom=2pt,
  oversize=2pt,
  colframe=white!75!black,
  before skip=\topsep,
  after skip=\topsep,
}

\newtheorem{Qns}{}[subsection]
\renewcommand{\theQns}{\arabic{Qns}}
\tcolorboxenvironment{Qns}{
  enhanced, frame hidden, borderline west = {1pt}{0pt}{blue!50},
  colback=blue!10,
  coltext=black,
  boxrule=1pt,
  boxsep=1pt,
  before skip=\topsep,
  after skip=\topsep,
}

\tcolorboxenvironment{proof}{
  enhanced, frame hidden, borderline west = {1pt}{0pt}{red!50},
  breakable,
  pad at break=2mm,
  colback=red!10,
  coltext=black,
  boxrule=1pt,
  boxsep=1pt,
  before skip=\topsep,
  after skip=\topsep,
}

\theoremstyle{plain}
\newtheorem{STheorem}{STheorem}
\tcolorboxenvironment{STheorem}{
  colback=green!15,
  boxrule=1pt,
  boxsep=1pt,
  left=2pt,right=2pt,top=2pt,bottom=2pt,
  oversize=2pt,
  colframe=blue!75!black,
  before skip=\topsep,
  after skip=\topsep,
}
\renewcommand{\theSTheorem}{\Roman{STheorem}}
\theoremstyle{remark}
\newtheorem*{fact}{Fact}
\tcolorboxenvironment{fact}{
  colback=white,
  boxrule=1pt,
  boxsep=1pt,
  left=2pt,right=2pt,top=2pt,bottom=2pt,
  oversize=2pt,
  colframe=white!75!black,
  before skip=\topsep,
  after skip=\topsep,
}
\newtheorem*{note}{Note}
\tcolorboxenvironment{note}{
  colback=white,
  boxrule=1pt,
  boxsep=1pt,
  left=2pt,right=2pt,top=2pt,bottom=2pt,
  oversize=2pt,
  colframe=white!75!black,
  before skip=\topsep,
  after skip=\topsep,
}
\newtcolorbox{IN}{
    colback=green!15,
    boxrule=1pt,
    boxsep=1pt,
    left=2pt,right=2pt,top=2pt,bottom=2pt,
    oversize=2pt,
    colframe=blue,
    before skip=\topsep,
    after skip=\topsep,
    title=Important Notes,
    colbacktitle=blue!75,
    coltitle=white,
    enhanced,
    attach boxed title to top left,
  }

\theoremstyle{plain}
\newtheorem{ZTheorem}{Theorem}[section]
\tcolorboxenvironment{ZTheorem}{
  colback=orange!15,
  boxrule=1pt,
  boxsep=1pt,
  left=2pt,right=2pt,top=2pt,bottom=2pt,
  oversize=2pt,
  colframe=blue!75!black,
  before skip=\topsep,
  after skip=\topsep,
}
\renewcommand{\theZTheorem}{\arabic{section}Z\Alph{ZTheorem}}
\newtheorem{ZLemma}[ZTheorem]{Lemma}
\tcolorboxenvironment{ZLemma}{
  colback=yellow!15,
  boxrule=1pt,
  boxsep=1pt,
  left=2pt,right=2pt,top=2pt,bottom=2pt,
  oversize=2pt,
  colframe=orange!15,
  before skip=\topsep,
  after skip=\topsep,
}
\newtheorem{ZCorollary}[ZTheorem]{Corollary}
\tcolorboxenvironment{ZCorollary}{
  colback=pink!25,
  boxrule=1pt,
  boxsep=1pt,
  left=2pt,right=2pt,top=2pt,bottom=2pt,
  oversize=2pt,
  colframe=orange!15,
  before skip=\topsep,
  after skip=\topsep,
}

\newtheorem{QTheorem}{Theorem}[section]
\tcolorboxenvironment{QTheorem}{
  colback=orange!15,
  boxrule=1pt,
  boxsep=1pt,
  left=2pt,right=2pt,top=2pt,bottom=2pt,
  oversize=2pt,
  colframe=blue!75!black,
  before skip=\topsep,
  after skip=\topsep,
}
\renewcommand{\theQTheorem}{\arabic{section}Q\Alph{QTheorem}}
\newtheorem{QLemma}[QTheorem]{Lemma}
\tcolorboxenvironment{QLemma}{
  colback=yellow!15,
  boxrule=1pt,
  boxsep=1pt,
  left=2pt,right=2pt,top=2pt,bottom=2pt,
  oversize=2pt,
  colframe=orange!15,
  before skip=\topsep,
  after skip=\topsep,
}
\newtheorem{QCorollary}[QTheorem]{Corollary}
\tcolorboxenvironment{QCorollary}{
  colback=pink!25,
  boxrule=1pt,
  boxsep=1pt,
  left=2pt,right=2pt,top=2pt,bottom=2pt,
  oversize=2pt,
  colframe=orange!15,
  before skip=\topsep,
  after skip=\topsep,
}

\newtheorem{RTheorem}{Theorem}[section]
\tcolorboxenvironment{RTheorem}{
  colback=orange!15,
  boxrule=1pt,
  boxsep=1pt,
  left=2pt,right=2pt,top=2pt,bottom=2pt,
  oversize=2pt,
  colframe=blue!75!black,
  before skip=\topsep,
  after skip=\topsep,
}
\renewcommand{\theRTheorem}{\arabic{section}R\Alph{RTheorem}}
\newtheorem{RLemma}[RTheorem]{Lemma}
\tcolorboxenvironment{RLemma}{
  colback=yellow!15,
  boxrule=1pt,
  boxsep=1pt,
  left=2pt,right=2pt,top=2pt,bottom=2pt,
  oversize=2pt,
  colframe=orange!15,
  before skip=\topsep,
  after skip=\topsep,
}
\newtheorem{RCorollary}[RTheorem]{Corollary}
\tcolorboxenvironment{RCorollary}{
  colback=pink!25,
  boxrule=1pt,
  boxsep=1pt,
  left=2pt,right=2pt,top=2pt,bottom=2pt,
  oversize=2pt,
  colframe=orange!15,
  before skip=\topsep,
  after skip=\topsep,
}

\newtheorem{innerPLemma}{PLemma}
\newenvironment{PLemma}[1]
  {\renewcommand\theinnerPLemma{#1}\innerPLemma}
  {\endinnerPLemma}
\tcolorboxenvironment{PLemma}{
  colback=white,
  boxrule=1pt,
  boxsep=1pt,
  left=2pt,right=2pt,top=2pt,bottom=2pt,
  oversize=2pt,
  colframe=black!50,
  before skip=\topsep,
  after skip=\topsep,
}

\usepackage{ stmaryrd }
\usepackage{ mathrsfs }
\usepackage{graphicx}
\graphicspath{ {./images/} }
\usepackage{enumitem}

\usepackage{fullpage}

\usepackage[pagestyles]{titlesec}
\titleformat{\chapter}[display]   
{\normalfont\huge\bfseries}{\chaptertitlename\ \thechapter}{20pt}{\Huge}   
\titlespacing*{\chapter}{0pt}{-50pt}{40pt}
\titleformat{\chapter}[display]{\normalfont\bfseries}{}{0pt}{\Huge}
\newpagestyle{mystyle}
{\sethead[\thepage][][\chaptertitle]{}{}{\thepage}}
\pagestyle{plain}

\renewcommand\thesection{\thechapter.\arabic{section}}
\renewcommand\thechapter{\arabic{chapter}}


\usepackage{geometry}
\geometry{
  bottom=15mm
}

\usepackage{hyperref}
\hypersetup{
    colorlinks=true,
    linkcolor=blue,
    filecolor=magenta,      
    urlcolor=cyan,
    }
\renewcommand{\thefootnote}{\arabic{footnote}}	
\usepackage{subfiles}
\begin{document}


\begin{tikzpicture}[font=\sffamily,remember picture,overlay]
    \node[fill=Goldenrod,anchor=north, minimum width=\paperwidth, minimum height=2cm] (names)
 at ([yshift=-3cm]current page.north) {
  \begin{tabular}{c}
      \\
      \color{blue} {\fontsize{24.88pt}{65pt}\selectfont A-Level Mathe}\\[2mm]
      \Large \color{blue}{Shao Hong}\\
      \\
  \end{tabular}
  };
\end{tikzpicture}

\begingroup
\let\clearpage\relax
\vspace{5cm}
\tableofcontents

\mainmatter
\endgroup

\newpage

\chapter{A1: Inequalities and Equations}
\section{Solving Inequalities}
\subsection{Rational Inequalities}
\begin{tcolorbox}[
    colback=yellow!20,
    boxrule=1pt,
    boxsep=1pt,
    left=2pt,right=2pt,top=2pt,bottom=2pt,
    oversize=2pt,
    colframe=blue!75!black,
    before skip=\topsep,
    after skip=\topsep,
    title=General Methods,
]
    \begin{enumerate}
        \item Quadratic formula for factorisation / finding roots (of polynomial).
        \item Completing the square.
        \item Discriminant to eliminate factors which are \emph{always} positive or negative (e.g. removing \(x^2-3x+4\)). \emph{Note to include coefficient of \(x^2\) in argument.}
        \item GC (include sketch).
        \item \emph{Rational Functions}\footnote{Fractions of Polynomials}: Move everything to one side (\(+\),\(-\)), then use number line.
        \item Number line (more complicated functions).
    \end{enumerate}
\end{tcolorbox}
\begin{IN}
    \begin{itemize}[label=\footnotesize \(\square\) \normalsize]
        \item Discriminant --- include coefficient of \(x^2\) in argument.
        \item Rational functions --- exclude the values that causes division by zero to occur. 
        \item Cross multiplication preserves order for \(\frac{x}{y}<\frac{x'}{y'}\) iff \(y\) and \(y'\) are \emph{both} positive or negative.\footnote{Otherwise, note the counterexample \(\frac{1}{2}<\frac{1}{-3}\).}
        \item Squaring preserves order for \(x<y\) iff \(x\) and \(y\) are \emph{both} positive or negative.
    \end{itemize}
\end{IN}
\section{Modulus Inequalities}
\begin{fact}
    Given \(x \in \mathbb{R}\), we have that 
    \begin{itemize}
        \item \(\lvert x \rvert \geq 0\),
        \item \(\lvert x^2 \rvert=\lvert x \rvert^2=x^2\),
        \item \(\sqrt{x^2}=\lvert x \rvert\).
    \end{itemize}
    And as long as \(x \in \mathbb{R}^+\),
    \begin{itemize}
        \item \(\sqrt{x}^2=\lvert x \rvert\).
    \end{itemize}
\end{fact}
\begin{tcolorbox}[
    colback=yellow!20,
    boxrule=1pt,
    boxsep=1pt,
    left=2pt,right=2pt,top=2pt,bottom=2pt,
    oversize=2pt,
    colframe=blue!75!black,
    before skip=\topsep,
    after skip=\topsep,
    title=Useful Properties,
]
    For every \(x,k \in \mathbb{R}\):
    \begin{enumerate}[label=(\alph*)]
        \item \(\lvert x \rvert < k\) iff\footnote{Notice that \(k>0\) here since \(0 \leq \lvert x \rvert < k\).} \(-k<x<k\).
        \item \(\lvert x \rvert > k\) iff \(x<-k\) or \(x>k\).
    \end{enumerate}
    \footnotesize (of course, similarly applies for the non-strict ordering \(\leq\)) \normalsize
\end{tcolorbox}
\begin{IN}
    \begin{itemize}[label=\(\circ\)]
        \item Note that when solving for \(\lvert x \rvert=y\), \(\lvert x \rvert < y\), etc, \(y\) must be greater than or equal to 0. In other words, there may be solutions we will need to reject.\\ \footnotesize(For \(<\), equality is of course not allowed.)\normalsize  
    \end{itemize}
\end{IN}


\end{document}